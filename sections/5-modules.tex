 \section{Data Types}
\label{sec:data_types}

The \sunpypkg package provides two core data types that are designed to provide a general, standard, and consistent interface for loading and representing solar data across different instruments and missions.
The core data types currently provided in \sunpypkg are \Timeseries and \Map, which support 1D temporal data and 2D image data, respectively. 
The purpose of these core classes is to accommodate the user with a standard data structure regardless of the data source (i.e. observational data from separate instruments). 
They maintain a consistent interface for accessing solar data attributes such as the data array itself as well as the meta data and relevant units. 
This allows for an easier workflow in the analysis of solar data observations.
These core classes also include functionality for data manipulation and data visualization. 
This section provides an overview of the \Timeseries and \Map data types.

\subsection{\Timeseries}
\label{sec:timeseries}
Many observations in the field of solar physics consist of time series data. 
For example, the X-ray Sensor aboard the Geostationary Operational Environmental Satellite (GOES), which is used as the classification standard for solar flares, continuously measures the solar integrated X-ray flux as a function of time in two broadband channels. 
The \Timeseries class in \sunpypkg aims to accommodate the necessary requirements to represent solar time series data by providing a self-consistent interface for users to interact with and manipulate time series data from a variety of solar instruments. 


\Timeseries holds both data as well as meta data and units data, which are stored as attributes. 
Users can create a \Timeseries either through local data source files (i.e. observational datasets) or through custom time series data.
A large amount of functionality for the manipulation of solar time series data is provided in \Timeseries. 
This includes the ability to add, update, truncate, re-sample and combine data within a \Timeseries or combine multiple \Timeseries together. 
\Timeseries also has its own built-in visualization methods to allow for easy inspection of the time series data. 

\begin{figure}
    \centering
    \includegraphics[width=0.55\textwidth]{figures/timeseries_example.pdf}
    \caption{An example of a GOES X-ray Sensor \Timeseries plotted over a 24 hour period. The two colors time series represent the two broadband channels of GOES 1--8~\AA\ (red) and 0.5--4~\AA\ (blue).  A solar flare is noted by the increase in flux and the grey dashed line denotes the flare peak as found from the Helio Event Knowledgebase (HEK).}
    \label{fig:timeseries_example}
\end{figure}

An example of a \Timeseries created from GOES X-ray sensor observations during a day for which a solar flare is present is shown in Figure~\ref{fig:timeseries_example}.
\Timeseries currently supports the data sources listed in Table \ref{tab:instruments} in addition to indices from the National Oceanic and Atmospheric (NOAA) Space Weather Prediction Center (SWPC) that track the solar cycle and its predicted progression. Due to its flexible data structure, it is easy to add additional instruments and data sources to the \Timeseries object.

%%%%%%%%%%%%%%%% TABLE %%%%%%%%%%%%%%%%%%%
\begin{table}
\begin{center}
\begin{tabular}{|p{12cm}|c|c|}
\hline
& \\
\textbf{Supported by \Timeseries}& \textbf{Instrument reference}\\
\hline
\hline
\textit{Geostationary Operational Environmental Satellite (GOES)} X-ray Sensor (XRS) & \citep{garcia94, hanser96} \\
\hline
\textit{Fermi} Gamma-ray Burst Monitor (GBM) &  \citep{meegan2009fermi} \\
\hline
\textit{Nobeyama Radioheliograph (NoRH)} & \citep{nakajima1994nobeyama} \\
\hline
\textit{PRoject for Onboard Autonomy (PROBA2)} Large Yield Radiometer (LYRA) & \citep{dominique2013lyra} \\
\hline
\textit{Solar Dynamics Observatory (SDO)} EUV Variability Experiment (EVE) & \citep{woods2010extreme}  \\
\hline
\textit{Reuven Ramaty High Energy Solar Spectroscopic Imager (RHESSI)} & \citep{lin2003reuven} \\
\hline
 & \\
\textbf{Supported by \Map} & \textbf{Instrument reference} \\
\hline
\hline
\textit{COronal Solar Magnetism Observatory (COSMO)} K-coronagraph (K-Cor) & \citep{dewijn12} \\
\hline
\textit{Hinode} X-Ray Telescope (XRT) & \citep{golub2008x}  \\
\hline
\textit{Interface Region Imaging Spectrograph (IRIS)} Slit Jaw Imager (SJI) & \citep{DePontieu2014}  \\
\hline
\textit{PRoject for Onboard Autonomy (PROBA2)} Sun Watcher using Active Pixel System detector and Image Processing (SWAP) & \citep{seaton2013swap} \\
\hline
\textit{Reuven Ramaty High Energy Solar Spectroscopic Imager (RHESSI)} & \citep{lin2003reuven} \\
\hline
\textit{Solar and Heliospheric Observatory (SOHO)} Extreme ultraviolet Imaging Telescope (EIT) & \citep{delaboudiniere1995eit}\\
\hline
\textit{Solar and Heliospheric Observatory (SOHO)} Large Angle Spectroscopic COronagraph (LASCO) & \citep{brueckner1995large} \\
\hline
\textit{Solar and Heliospheric Observatory (SOHO)} Michelson Doppler Imager (MDI) & \citep{scherrer1995solar}\\
\hline
\textit{Solar Dynamics Observatory (SDO)} Atmospheric Imaging Assembly (AIA) & \citep{lemen2012} \\
\hline
\textit{Solar Dynamics Observatory (SDO)} Helioseismic and Magnetic Imager (HMI) & \citep{schou12}  \\
\hline
\textit{Solar TErrestrial RElations Observatory (STEREO)} Extreme Ultraviolet Imager (EUVI), COronagraph 1 and 2 (COR1/2) for both \textit{STEREO} A and B & \citep{howard2008sun} \\
\hline
\textit{Transition Region and Coronal Explorer (TRACE)}  & \citep{handy99}  \\
\hline
\textit{Yohkoh} Soft X-ray Telescope (SXT) & \citep{tsuneta1991soft}  \\
\hline
\end{tabular}
\end{center}
\caption{The following table outlines the instruments supported by the \Timeseries and \Map objects described in Section \ref{sec:data_types}.}
\label{tab:instruments}
\end{table}
%%%%%%%%%%%%%%%% TABLE %%%%%%%%%%%%%%%%%%%

\subsection{\Map}
\label{sec:map}
Images of the Sun and the heliosphere are an important data type in solar observations. 
For example, the Helioseismic and Magnetic Imager (HMI) instrument aboard the Solar Dynamics Observatory (SDO) maps the magnetic field at the solar photosphere every 45 seconds with 4k $\times$ 4k pixel resolution. 
These high resolution image sets require precise coordinate information in order to achieve much solar science, particularly when performing multi-instrument studies. 

The \Map class in \sunpypkg provides a framework to contain and analyze 2D image data that have an associated coordinate frame and relevant meta data. 
A \Map can be created by providing input data files located locally or fetched via the \sunpypkg data search and retrieval interface \Fido (see Section \ref{sec:fido}).
The \Map class will automatically detect the instrument of the data source and subsequently search the specific meta data to infer the coordinate system from the appropriate FITS keywords \citep{refId0, 2006A&A...449..791T}. Other source-specific meta data is also loaded into the \Map class such as color tables and appropriate image scaling for each instrument.


The \Map object permits users to plot not only a single image but also overlay multiple images; such functionality is useful in displaying data from instruments with overlapping fields of view. The \Map object also loads source-specific color
Users can also combine multiple maps together in a time-ordered sequence. The data in each \Map need not have the same size (number of pixels in each direction) or view the same area of the sky (field-of-view). 

\begin{figure}
    \centering
    \includegraphics[width=0.97\textwidth]{figures/map_example.pdf}
    \caption{An example of a \sunpypkg \Map plotted from observations of the 171~\AA\ wavelength channel of the Atmospheric Imaging Assembly (AIA) aboard the Solar Dynamics Observatory (SDO).The left hand panel shows a full disk image of the Sun whereas the right hand panel shows a zoom in of the white box in the left hand panel, focusing on the flare that erupted (the same event as the \Timeseries plot in Figure~\ref{fig:timeseries_example}).}
    \label{fig:map_example}
\end{figure}

\Map currently supports the data sources listed in Table \ref{tab:instruments} as well as the Helioviewer JPEG2000 image files of for each of these data sources. Similar to \Timeseries, it is easy to add additional instruments and data sources to \Map given the flexibility in the data type structure. 
