\section{Affiliated Packages - Steven}
\label{sec:affil_package}
% edited by Steven C. 23-Apr-2019

In order to foster collaboration and code re-use, the \sunpyproj supports the concept of affiliated packages. 
These are \python package that build upon the functionality of the \sunpypkg package or provides general functionality useful to solar physics. 
Affiliated packages can also be used to develop and mature subpackage functionality outside of the constraints of \sunpypkg. 
The following requirements must be satisfied by any potential affiliated packages.
\begin{itemize}
    \item the package must make use of all appropriate features in \sunpypkg.
    \item Documentation must be provided that explains the function and use of the package, and it should be of comparable quality to \sunpypkg.
    \item A test suite must be provided to verify the correct operation of the package.
\end{itemize}
Developers can formally apply to become an affiliated package to the lead developer. 
Final approval is required by the \sunpy board for acceptance. 
Packages are re-reviewed on a yearly basis to ensure that they continue to meet the standards. 
All affiliated packages are listed on \url{sunpy.org}, are provided support by the \sunpy developer community, and advertised at conferences and workshops. 
In addition, the project further defines sponsored affiliated package which is an affiliated package whose maintenance and development is the responsibility of the \sunpy project.  

Add a line about providing a package template for affiliated packages.

The following sections provides short descriptions of the existing affiliated packages.
\label{sec:affil_packages}

\subsection{drms - Kolja Glogowski - MONICA}
\label{sec:drms}
\todo{reviewed by Steve Christe - 26-mar-2019}

The \package{drms} affiliated package provides functionality which allows for direct access to data hosted by the Joint Science Operations Center (JSOC). 
This is the primary data center for the Solar Dynamics Observatory’s (SDO)\todo{check if these instruments have already been reference before this section, and make sure that reference are provided} Helioseismic and Magnetic Imager (HMI) and Atmospheric Imaging Assembly (AIA) instruments as well as data from the Solar and Heliospheric Observatory's (SoHO) Michelson Doppler Imager (MDI) instrument. 
DRMS stands for Data Record Management System, a pSQL database that contains metadata, as well as pointers to image data, for every image taken by AIA, HMI, and MDI. 
The \package{drms} package provides access to the unique search capabilities of the JSOC which include: metadata search queries, export tailored FITS files and serve these files in a variety of methods, as well as export data as movies and images in various formats.

\todo{schriste - clean up, add more detail on that last sentence, also say what is the JSOC API interface}.

\subsection{ndcube - Dan Ryan}
\label{sec:ndcube}
% edited by Steven C. 25-Apr-2019

The \package{ndcube} package provides functionality for manipulating N-dimensional coordinate-aware data.  
It can support any combination of axis-types for example
images (2 spatial axis), images over time (2 spatial and 1 time axis), spectrograms
(wavelength and time), as well as more complex data sets such as from slit spectragraphs (wavelength, time, and spatial). 
It provides the class \sunpycode{NDCube} which is a
subclass of \astropy's \sunpycode{NDData} data container class which hold together
the data array uncertainties, and (potentially) a data mask. 
\sunpycode{NDCube} adds
support for handling world coordinate transformations through the World Coordinate System (WCS) architecture commonly used in solar physics (\todo{insert WCS reference here}). 
This package provides powerful and intuitive slicing that allow users to slice the entire dataset with a single command using either array indices or real world coordinates. 
This slices all components including the data array, and coordinates simultaneously. 
This enables users to manipulate their dataset more quickly and accurately, allowing them to more efficiently and reliably achieve their science goals and is meant to be used as a basis for more advanced and instrument specific functionality (see Section~\ref{sec:irispy}). 
Support for generalized WCS module (\todo{insert reference here}) is planned to be added in the next major release.

\subsection{radiospectra - David}


\subsection{IRISPy - Dan Ryan}
\label{sec:irispy}
% edited by Steven C. 25-Apr-2019

\package{IRISPy} provides tools to read, manipulate and visualize data from the Interface Region Imaging Spectrograph (IRIS; \citealt{DePontieu2014}). 
IRIS is a NASA Small Explorer mission which has two instruments; a slit-jaw imager (SJI) and a rastering slit spectrograph (SG). 
The \package{IRISPy} is limited to reading level 2 data for either of these instruments. 
This package provide data classes \sunpycode{IRISMapCube} and {IRISSpectrogramCubeSequence} which hold data from SJI and SG respectively.
Built on top of the functionality provided by \package{ndcube} (see Section\~ref{sec:ndcube}) they link the main observations, metadata, uncertainties, data unit, mask, and WCS transformations and provide easy slicing of the data in any axis. 
Measurement uncertainties accounting for Poisson statistics and readout noise are automatically calculated while a mask identifies what pixels of the data array are exposed to the Sun. 
The mask facilitates basic operations, e.g.\ mean, max, etc., and removes bad pixels from the color table in visualizations.


%This is useful when one axis represents two coordinates, one of which is parallel while the other is not.
%For example, a scanning slit-spectrograph can scan its slit through a number of positions along the x-axis, before moving back to the original position and repeating the process.
%Therefore the sequence axis is parallel, as the scans are sequential in time, but also perpendicular because in that the scan number represented by the sequence axis is not related to the spatial position of the slit represented by the x-axis of the constituent \sunpycode{NDCubes}. 
%Users may want to think of the x-axis as time, in which case the \sunpycode{NDCubeSequence} is like an ND dataset, then instantly change to thinking about the x-axis as slit position, in which case the \sunpycode{NDCubeSequence} is better represented as an N+1D dataset.
%To achieve this model in the past, the data would have to be duplicated and placed into two separate objects with different shapes and sets of coordinate transformations.
%This paradigm requires both objects be kept consistent, leading to a greater workload and potentially more errors.
%By contrast, \sunpycode{NDCubeSequence} makes this possible for a single object, speeding up analysis, minimizing memory resources, and reducing the potential for errors.

%Moreover, because the data unit is linked to the object, it is always obvious what unit the data is in. This saves scientists the hassle of performing important, but laborious and repetitive data conversions and avoid confusion by always tracking the unit of the data through those conversions. This leads to more efficient and accurate science.