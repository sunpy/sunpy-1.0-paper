\section{Physical Units - Bobra}
\label{sec:units}
\todo{edits by Steven Christe}

%Discuss the units SEP. discuss that we are using astropy units and provide a short description of it's capabilities. mention that sunpy.sun provides quantities with units. mention the NASA units disaster and reference units policy for NASA as reference here is a interesting reference \url{https://sma.nasa.gov/news/safety-messages/safety-message-item/lost-in-translation}

Calculations of physical quantities have traditionally been performed in software 
using bare numbers in the interest of speed as well as simplicity. Physical units
have frequently been recorded in comments or in documentation which can easily lead
to errors and potentially catastrophy. The Mars Climate Orbiter mishap in 1988 was caused by a unit discrepancy. The spacecraft trajectory was reported in English units instead metric which led to the the Mars Climate Orbiter entering the Martian atmosphere well-below its intended altitude causing complete mission failure \citep{mco_mishap_report}. A more modern approach is described by \citep{Damevski2009}
which consists of describing units at the system architecture level. This capability 
is provided by the \package{astropy.units} module and and a SEP mandates its use.

The \package{astropy.units} module defines a \code{astropy.units.Quantity} class which combine a value and a unit. This class is an extension to \code{numpy.array} and provides significant performance compared to other approaches \todo{add a reference here?}. Every input and output of \sunpypkg functions that requires a physical value make use of \code{astropy.units.Quantity}. Enforcement on input is provided by a decorator (\code{quantity\_input()}). This approach eliminates any confusion about the units of a physical quantity in input or output. Also conversions between such as Gauss and Tesla units are made straightforward with the \code{.to()} method.

The \package{sunpy.sun} module contains constants, parameters and models of the Sun provided as \code{astropy.units.Constants} which are \code{astropy.units.Quantity} with additional reference metadata. These include variable quantities, such as the Carrington rotation number, as well as constants, such as the solar mass. 

\todo{do we want to discuss astropy time in this section as well?}

