\section{Data Types - laura}
\todo{ready for review}
\label{sec:data_types}
\todo{How many of the operating Heliophysics observatory data products do we support? https://www.nasa.gov/content/goddard/heliophysics-system-observatory-hso}
The \sunpypkg package provides core data types that are designed to provide a standardized interface to data structures across data types (images, lightcurves, spectra) as well as data sources. 
These core data types currently provided by \sunpypkg are \Map and \Timeseries classes which support 2D image data and 1D temporal data, respectively. 
They offer a consistent interface to the user allowing a simpler work-flow in the analysis and manipulation of observations. 
These objects provide visualization and basic manipulation routines with a consistent API. 
For example, metadata is provided by the \code{.meta} property, the data is stored in \code{.data}. 
They also handle all of the manipulation necessary to read data in from mission-specific files. 
The sections provide an overview of the \Timeseries and \Map objects. 

\subsection{\Timeseries - data with one temporal dimension}
\label{sec:timeseries}
\todo{ready for review}
\todo{reviewed by Steven - 22-mar-2019}
Time-series data is a fundamental observational data type. 
The \Timeseries class is designed to handle time-series data through a robust and consistent interface. 
The inherent structure of a \Timeseries object consists of times and measurements while the underlying structure used to store the data is a \code{pandas.DataFrame}. 
It supports time-series data from a wide range of solar instruments. 

The \GenericTimeSeries class is the base class of \Timeseries, which is created through the \Timeseries factory. 
A number of instruments are supported through subclassing which have instrument-specific methods for reading source files. 
Custom \Timeseries can also be made from input data in the form of a \code{pandas.DataFrame}, an \code{astropy.table.Table} or a \package{numpy} array. 

\Timeseries currently supports data sources from the following instruments: the Geostationary Operational Environmental Satellite (\textit{GOES}) X-ray Sensor (XRS), \textit{SDO} EUV Variability Experiment (EVE) \citep{woods2010extreme}, \textit{PROBA2} Large Yield Radiometer (LYRA) \citep{dominique2013lyra}, \textit{Fermi} Gamma-ray Burst (GBM) monitor \citep{meegan2009fermi}, the Nobeyama
Radioheliograph (\textit{NoRH}) \citep{nakajima1994nobeyama}, and \textit{RHESSI} \citep{lin2003reuven}. 
The \Timeseries\ object also supports the National Oceanic and Atmospheric (NOAA) Space Weather Prediction Center (SWPC) solar cycle monthly indices and predicted progression. 
These data sources are supported through a \Timeseries\ source file for each listed above. 
With this structure additional instruments and data sources can easily be added. 
\Timeseries holds meta data, stored in the \Timeseriesmetadata object. 
This functionality is designed to allow the user to create a single \Timeseries by combining multiple \Timeseries together into one, preserving the metadata relevant to each cell, column or row, concatenated into an organized fashion. 

\Timeseries also supports manipulation functionality for working with time-series data including adding new columns of data to a \Timeseries, truncating a \Timeseries over a specified time range, resampling, and creating other data products from an existing \Timeseries, such as into a \code{pandas.Dataframe} or an \code{astropy.table}. 
The \Timeseries object, similar to \Map (Section \ref{sec:map}), has it's own visualization plotting methods allowing for easy inspection of the data.

\subsection{Map - data with two spatial dimensions}
\label{sec:map}
\todo{ready for review}
The \Map class provides the functionality to store 2D data associated with a coordinate system and relevant metadata. 
A \Map object is created using the \Map factory which will produce a \GenericMap object or a subclass of \GenericMap which deals with instrument specific data.  
The main use of \Map is to store and manipulate images of the Sun and heliosphere.

A number of instruments are explicitly supported in \sunpypkg, defined within the subclass source files for each instrument source. 
The source file provides a compatibility layer which converts the specific meta data and other source-specific parameters to the standard \GenericMap\ interface. 
This also includes properties such as source-specific color tables and appropriate image scaling provided by the instrument teams. 
\sunpypkg currently supports the following instruments as part of the \GenericMap;
\begin{inparaitem}
\item \textit{SDO} - Atmospheric Imaging Assembly (AIA) \citep{lemen2011atmospheric} and the Helioseismic and Magnetic Imager (HMI) \citep{scherrer2012helioseismic}. 
\item \textit{SOHO} - Large Angle Spectroscopic COronagraph (LASCO) \citep{brueckner1995large}, Extreme ultraviolet Imaging Telescope (EIT) \citep{delaboudiniere1995eit}, and Michelson Doppler Imager (MDI) \citep{scherrer1995solar}
\item \textit{STEREO} - Extreme Ultraviolet Imager (EUVI), COronagraph 1 and 2 (COR1/2) for both \textit{STEREO} A and B \citep{howard2008sun}
\item \textit{Hinode} - X-Ray Telescope (XRT) \citep{golub2008x}
\item \textit{IRIS} Slit Jaw Imager (SJI) \citep{DePontieu2014}
\item \textit{COronal Solar Magnetism Observatory (COSMO)} -  K-coronagraph (K-Cor) all polarized brightness
\item \textit{PROBA2} - Sun Watcher using Active Pixel System detector and Image Processing (SWAP) \citep{seaton2013swap}
\item \textit{RHESSI} \citep{lin2002reuven}
\item \textit{TRACE}
\item \textit{Yohkoh} Soft X-ray Telescope (SXT) \citep{tsuneta1991soft}.
\end{inparaitem}
Helioviewer JPEG2000 image files of the above data sources are also supported by the \Map\ class.

A \sunpy \Map\ can be created either from data files from which the \Map\ factory will automatically detect the type of file, associated instrument and search the appropriate FITS keywords to infer the coordinate system. A custom \GenericMap can also be created by providing the \Map\ object with data and basic meta information.
Two additional objects based on \Map\ are also provided in \sunpy. 
The \code{MapSequence} object stores a time-ordered sequence of \Map objects.  
The data in each \Map need not have the same size (number of pixels in each direction) or view the same area of sky (field-of-view). \code{MapSequence} is useful for examining timeseries of solar images.  
The \code{CompositeMap} object permits the simple overlay and plot multiple \Map objects; such functionality is useful in displaying data from instruments with overlapping fields of view.


