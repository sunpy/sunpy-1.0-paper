\subsection{Units and Time Scales}
\label{sec:units}

Solar observations are composed of a physical quantity measured at a specific time.
Specifying physical quantities and time accurately and precisely are therefore fundamental to any solar data analysis task.
In recognition of this, two SEPs have been written to describe the problem and each mandates a specific solution.

Historically, calculations using physical quantities have been performed in software using raw numbers with no associated units.
At best, the unit information might be provided in a comment or encoded in a variable name.
However, this separation between numbers and units can lead to calculation errors and potentially lead to disaster\footnote{As an extreme example, the Mars Climate orbiter mission in 1988 failed due to a unit discrepancy \citep{mco_mishap_report}.}.
A potential solutions to the problem of dimensional errors was described by \citet{Damevski2009} which advocated for the solution to be provided at the software architecture layer.
\sunpypkg utilizes the unit-aware functionality provided by the \package{astropy.units} subpackage throughout the entire code base.
This package provides support for physical quantities through a \code{Quantity} class, which consists of a number and its associated unit(s).
These quantities can be combined in expressions with unit conversions and cancellations automatically taken into account.
Tests have shown that the performance overhead incurred by this functionality is typically minimal.

SEP-0003 \citep{sep-0003} formally mandates that all user-facing functionality provided by \sunpypkg make use of \package{astropy.units}.
All functions have their input constrained to the appropriate type of unit (e.g. length, mass) and provide an error if the input is not correct.
Input can then be provided with any appropriate units (e.g. mm, km, inches) and conversions occur automatically without user intervention.
The \package{sunpy.sun.constants} subpackage contains many standard constants relevant to solar physics with additional information such as uncertainty and reference.

Time is another fundamental quantity that must be appropriately specified for scientific uses.
Solar data analysis requires certain functionality to manipulate and represent times and dates, such as having support for leap seconds and the ability to represent specific time scales and formats. 
This requirement goes beyond that provided by the standard \python \package{datetime} subpackage.
For this reason, SEP-0008 \citep{sep-0008} was mandated to adopt the use of the \package{astropy.time} subpackage to represent a date or time throughout the \sunpypkg code base.
The \package{astropy.time} provides the necessary support non-UTC time scales such as International Atomic Time (TAI), used for some solar data observations (e.g. HMI), and also provides functionality for representing time formats used in solar physics such as Julian day and Unix time. 
In the same manner as pairing numbers with their units, having a time class with this functionality pairs a time representation with its time scale, and allows for conversions to other time scales. 
This is particularly useful when performing multi-instrument studies (from which different observational datasets have different time formats). 
Furthermore the adoption of the \package{astropy.time} within \sunpypkg provides the necessary support for leap seconds, light travel calculations, and high precision time representation. 



