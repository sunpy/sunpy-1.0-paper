\section{Introduction}
\label{sec:intro}

Research astrophysicists rely on software for the analysis of data which is ever-growing and increasingly complex.
Scientists that study our nearest star, the Sun, are no different.
Solar physics relies (mostly) on remote sensing to measure the properties of our star.

In the past, most solar data analysis was performed using Fortran, a general-purpose, compiled programming language designed for scientific and engineering applications.
In the 1980s, the solar community made the transition from Fortran to IDL (Interactive Data Language), a commercial and closed-source interpretive programming language.
The community extracted several significant benefits from this transition.
Two significant ones are that IDL is an interpretive language which enables much faster (and therefore cheaper) development and access to a large number of powerful libraries that support scientific data analysis.
The astronomy community made the same transition around the same time which led to the release of the IDL Astronomy User's Library which allowed for significant code re-use.
Since the transition, significant and powerful functionality has been developed as part of Solarsoft, an open-source library of integrated software libraries that provide a common data analysis environment for solar physics \citet{freeland1998}.

A transition to Python will provide an even greater leap in capabilities to the solar community.
Astronomy is already making this transition (cite astropy).
Python is an interactive, interpretive, object-oriented, portable and high-level programming language which is extensible with compiled code in lower-level languages.
The strengths of the scientific Python environment have been enumerated in a number of other papers (cite cite cite). These include (add strengths here).
In relation to other scientific communities, the solar community is relatively small.
In order to keep up with increasingly difficult scientific challenges and produce world-class science, those features of Python that are most relevant are those that allow the community to leverage existing external resources.

Python is now used by most universities to teach computer science.
It is now the case that 8 of the top 10 CS departments (80\%), and 27 of the top 39 (70\%), teach Python in introductory computer science courses (Guo 2014).
A transition to Python means that new members of the solar community will not need to learn a new language to become productive researchers significantly reducing the training burden on the community and accelerating the path to scientific results.
Python is also one of the most widely-used computer programming languages.
It therefore provides access to a world-wide community of developers which can help solve problems for the solar community both directly and indirectly.
For example, many of the most common programming questions have already been asked and answered on websites such as Stackoverflow.
Hiring developers with Python experience is straightforward and open source projects can more easily recruit enthusiasts to contribute.
The scientific Python environment provides access to a rich ecosystem of specialized tools.
These include tools which provide advanced image processing techniques (scikit-image, ), machine-learning (scikit-learn, ), Bayesian analysis (), accelerated/parallel computing (dask, CuPy).
Finally, cloud-based computing infrastructure provides access to powerful and scalable computing resources (e.g. Azure, AWS, Travis) sometimes at no cost. Because Python is free and widely used, it is straightfoward to run multiple Python processes with no limitations from potentially costly licenses.

The \sunpyproj was created to bring these benefits to the solar community.
The mission of the \sunpyproj is to facilitate and promote the use and development of community-led, free, and open source\footnote{\url{https://opensource.org/osd}} solar data analysis software packages based on the scientific \python\footnote{\url{https://www.python.org/}} environment.
To achieve this goal, the project develops and maintains a core package (\sunpypkg) and supports an ecosystem of affiliated packages (see \autoref{sec:affil_package}) consistent with best practices \citep{Wilson:2014cka}

The \sunpyproj was officially founded in March of 2014 to manage the already developing \sunpypkg package.
The project selected the \python programming language to leverage the rich ecosystem of packages already available for general data analysis.
These include \numpy for multi-dimensional array manipulation \citep{numpy}, \scipy for scientific functions \citep{scipy}, \matplotlib for publication-quality 2D plotting \citep{matplotlib}, and \pandas for data structures and time series analysis \citep{pandas}.
These core packages form the backbone of hundreds of additional scientific \python packages.
Of particular relevance to solar physics is the \astropypkg package, which provides core functionality for the analysis of astrophysical data \citep{astropy2018}.

This paper describes the first stable release (v1.0) of the core package.
A previous paper describes v0.5 \citep{Community:2015cy}.
This article is not meant to replace the \sunpypkg documentation but provides an overview of the organization and highlights important functionality.
The full text of the paper, including all of the code to produce the figures, is available in a \github repository\footnote{\url{https://github.com/sunpy/sunpy-1.0-paper}}.
