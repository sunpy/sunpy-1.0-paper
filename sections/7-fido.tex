\section{The SunPy Core Package}
\label{sec:sunpycore}

The aim of the \sunpypkg package, the core package in SunPy ecosystem, is to provide a set of Python tools for performing common tasks in the analysis of solar data. These include querying and downloading data, loading and visualizing time series and images, and performing coordinate transformations. In the following sections, we describe the primary capabilities of \sunpypkg and highlight several improvements to the package that are included in version 1.0.

\subsection{Data Search and Retrieval}
\label{sec:fido}

One of the most important tasks that must occur before any analysis can take place is to search for and retrieve data.
A particular science goal may require data from multiple data providers, each of which may have different methods for data search and retrieval.
This heterogeneity increases the effort required by scientists to get the data they need.
In order to address this issue, the \package{sunpy.net} subpackage provides interfaces to many commonly used data providers and catalogues in solar physics.

\subsubsection{The \Fido Interface}
\label{sec:fido}

The most mature and powerful component of \package{sunpy.net} is the \Fido interface for data search and retrieval.
\Fido provides a unified interface that simplifies and homogenizes search and retrieval by allowing data to be queried and downloaded from multiple solar sources simultaneously irrespective of the underlying client.
Currently \Fido supports the Virtual Solar Observatory (VSO), the Joint Science Operations Center (JSOC, see \autoref{sec:drms}) and a number of individual data providers that make their data available via web-accessible resources such as HTTP websites (RHESSI, SDO-EVE, NOAA GOES soft X-ray flux, PROBA2-LYRA and NOAA sunspot number prediction) and FTP servers (NOAA sunspot number, Nobeyama Radioheliograph).

A \Fido search can include multiple instruments, and can query all available data providers with a single query.
Search queries can be made up of many different attributes such as instrument, time range, and wavelength.
The attributes can be joined using Boolean operators to enable complex queries.
The result of a query can be inspected and edited before retrieval.
The result of the \Fido search query is downloaded via asynchronous and parallel download streams.
\Fido also recognizes failed data downloads and allows for re-requesting files which were not retrieved.

\subsubsection{HEK Client}
\label{sec:hek}

In addition to data download, access to event catalogues are also an important aspect of solar physics research.
The primary solar event catalog is the Heliophysics Event Knowledgebase \citep[HEK,][]{hek} which provides a searchable database of manually and automatically detected solar features and events such as sunspots, solar flares, coronal mass ejections, etc. \sunpypkg provides a HEK search client which is highly flexible, allowing multiple event types and their properties to be queried simultaneously.
For example, it is possible to search for SPoCA \citep{2014AA...561A..29V} active regions above a user-specified size within a given time-range.

\subsubsection{Helioviewer Client}
\label{sec:helioviewer}

Finally, \package{sunpy.net} has a Helioviewer\footnote{\url{https://helioviewer.org/}} client which permits the user to query the Helioviewer JPEG2000 image archive.
Helioviewer provides a powerful online solar image browsing tool and the helioviewer client provides access to the browse data archive used by tool.
It can download image data and easily construct images of solar data from multiple sources.
It is currently planned for this functionality to be moved to an affiliated package.
